\chapter{Considerações Finais}

Este trabalho propõe um sistema capaz de monitorar remotamente a utilização da água em uma residência. Os sistemas de monitoramento propostos foram implementados. Como mostrado, a implementação é capaz de cadastrar, editar e autorizar usuários. Monitorar o fluxo de água e o tempo de banhos com gráficos e estados de sensores em tempo real remotamente, comportando-se como deveria.

O sistema físico não foi criado, devido à necessidade de proteção extra, como caixas resistentes à água e vapores para garantir o seu funcionamento e durabilidade.

Ao final do desenvolvimento do projeto e dos resultados apresentados, pode-se concluir que o sistema é eficaz no monitoramento remoto auxiliando o usuário na gestão da utilização de água em chuveiros. Podendo ser um grande aliado na redução do consumo de água em toda a residência, se implementado em outros setores, como torneiras ou tanques.

Para trabalhos futuros, pode-se realizar a implementação física em chuveiros com sensores e proteção mais robustos que sejam resistentes ao vapor de água presente ao se tomar banho. 

Seria interessante a implementação de mais sistemas de monitoramento utilizando outros sensores realizando a integração com HomeAssistant implementando apenas pequenas modificações e configurações extras. O sistema foi arquitetado e construído com o intuito de facilitar esta inclusão de sensores e reutilização.