% ----------------------------------------------------------
% Introdução (exemplo de capítulo sem numeração, mas presente no Sumário)
% ----------------------------------------------------------
\chapter[Introdução]{Introdução}
%\addcontentsline{toc}{chapter}{Introdução}
% ----------------------------------------------------------

A automação de casas e prédios, conhecida como domótica, é uma área que vem crescendo cada vez mais. Diretamente ligada ao controle e automação de residências, seus objetivos fundamentais são de oferecer conforto, facilidade de acesso, controle e segurança.\cite{teza2002alguns}

Setenta por cento da superfície do planeta é coberta por água, quase toda salgada e, portanto, imprópria para o consumo humano. Apenas 2,5\% desse total é potável e a maior parte das reservas (cerca de 80\%) está concentrada em geleiras nas calotas polares. 
Essa quantidade mínima de recursos aliada ao contínuo e intenso crescimento demográfico ao longo dos anos, o desenvolvimento industrial e, por consequência, o aumento do consumo de água nas grandes cidades, tem sido um dos principais temas de discussões e palestras de conscientização por todo o mundo.\cite{aguaconsumo}

Um assunto recorrente que há muito deixou de ser restrito às regiões áridas e desérticas com baixa disponibilidade de água per capita, faz com que governos e organizações de todo o mundo estejam com atenções voltadas para a criação de políticas de consumo sustentável, programas de educação ambiental, alternativas e soluções para a redução e controle do uso da água.\cite{ferreirasistema} 

Portanto, esse trabalho apresenta um protótipo capaz de adquirir dados em um ponto de consumo escolhido pelo usuário e apresentar essas informações através de gráficos em um dispositivo móvel.


\section{Justificativas e Relev{\^a}ncia}
%
Ao passar dos anos, os desperdícios da água utilizada atingem níveis nunca imaginados \cite{rebouccas2003agua}. Ao se juntar o interesse e conhecimento em eletrônica e automação, será possível otimizar o monitoramento do gasto de água em residências e prédios, visando coletar dados, para talvez uma diminuição deste desperdício. Esta foi a motivação inicial para a realização deste projeto e documento.
%
\section{Metodologia}

Será desenvolvido um projeto utilizando Raspberry Pi 3, ESP8266 e sensor de fluxo de água YF-S201.

O projeto deverá conectar os componentes via wi-fi, o YF-201 conectará com o ESP8266 que comucará com o RasperryPi, que processará os dados recebidos pelo microprocessador, guardando estes em um banco de dados que ficará na nuvem.

Um sistema para acompanhamento remoto dos dados recebidos pelo sensor também será implementado para facilitar a análise futura dos dados.

\section{Objetivos}

Desenvolver um sistema modular para monitoramento e controle de consumo de água através da integração entre microprocessadores, sensores e microcontroladores.

Este sistema será capaz de armazenar dados vindos dos sensores de fluxo de água em bancos de dados instalados em sistemas remotos.

Os dispositivos do projeto irão utilizar redes wi-fi para comunicar entre si, aumentando assim a portabilidade e custo com cabeamento do sistema.