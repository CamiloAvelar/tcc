% ----------------------------------------------------------
% Introdução (exemplo de capítulo sem numeração, mas presente no Sumário)
% ----------------------------------------------------------
\chapter[Introdução]{Introdução}
%\addcontentsline{toc}{chapter}{Introdução}
% ----------------------------------------------------------

Cerca de 70\% da superfície do planeta é coberta por água, quase toda salgada e, portanto, imprópria para o consumo humano. Apenas 2,5\% desse total é potável e a maior parte das reservas (cerca de 80\%) está concentrada em geleiras nas calotas polares. 
Essa quantidade reduzida de recursos aliada ao contínuo e intenso crescimento demográfico ao longo dos anos, o desenvolvimento industrial e, por consequência, o aumento do consumo de água nas grandes cidades, tem sido um dos principais temas de discussões e palestras de conscientização por todo o mundo. \cite{aguaconsumo}

Pesquisas indicam que, em poucas décadas, as reservas de água doce do planeta não serão suficientes para suprir as necessidades humanas caso os níveis de consumo não sejam controlados desde já \cite{Diarias2007}. A escassez deste recurso essencial à vida acarretará em problemas de ordem política, econômica e sanitária, podendo até originar conflitos similares aos causados pelo domínio do petróleo.

A economia de água é um assunto recorrente que há muito deixou de ser restrito às regiões áridas e desérticas com baixa disponibilidade de água per capita. Isto faz com que governos e organizações de todo o mundo estejam com atenções voltadas para a criação de políticas de consumo sustentável, programas de educação ambiental, alternativas e soluções para a redução e controle do uso da água. \cite{ferreirasistema}.

A fim de evitar consequências como a escassez da água, o consumo
responsável encabeça a lista de medidas a serem tomadas por se tratar de uma atitude factível a todas as pessoas. \cite{Diarias2007}. Recentemente, avanços em recursos computacionais e tecnologias de eletrônicos permitiram a criação do paradigma do IoT (Internet of Things ou Internet das Coisas). \citeauthor{Perumal2016} (\citeyear{Perumal2016}) descrevem a Internet das Coisas como sendo um método para conectar coisas em torno do ambiente e realizar um certo tipo de troca de mensagem entre eles, integrando-os.

O IoT representa uma rede mundial de objetos interconectados e unicamente endereçados. Esta conectividade entre sensores e atuadores permite o compartilhamento de informações entre plataformas através de um \textit{framework} unificado, desenvolvendo uma comum capacidade de criar aplicações inovadoras. Isto é possível devido a sensores, analise da dados e representação de informações através de Computação em Nuvem como o \textit{framework} unificado. \cite{RisteskaStojkoska2017}

Uma das grandes influências do IoT é no campo do monitoramento do ambiente em que vivemos, sistemas de alarmes e análise de dados ambientais \cite{Perumal2016}. %Este trabalho propõe a elaboração, planejamento e implementação de um sistema de baixo custo para monitoramento e economia de água no uso de chuveiros elétricos residenciais.

O desenvolvimento de tecnologias de infraestrutura no início do século XX, como as redes de água e esgoto, gás encanado e eletricidade fizeram com que a residências se conectassem com o meio externo, tornando-se um nó de uma grande rede \cite{forty2007objetos}. Com o advento da Internet, essa ligação se acentuou, permitindo ainda mais conectividade. \cite{VarelaDeSouza} 

Segundo \citeauthor{VarelaDeSouza} (\citeyear{VarelaDeSouza}), a palavra "Domótica" resulta da junção da palavra latina
“Domus”, que significa casa, com “Robótica”, que pode ser entendido como controle automatizado de algum processo ou equipamento. Seu uso pode trazer significativas vantagens a seus usuários como a otimização e gestão de recursos, praticidade, segurança, controle e monitoramento remoto dos dispositivos automatizados.



\section{Objetivos}

O objetivo deste trabalho é desenvolver um sistema modular, de baixo custo, baseado em microsserviços e no paradigma do IoT, para monitoramento e controle de consumo de água de chuveiros elétricos através da integração entre microprocessadores, microcontroladores, sensores e atuadores. 

O sistema será capaz de armazenar e exibir dados vindos dos sensores de fluxo de água em sistemas remotos, conectados através da rede Wi-Fi, além de comandar um atuador para interromper o fornecimento da água.

Será criado também um sistema que emulará os dados provindos do sensor, atuador e teclado utilizados.

\subsection{Objetivos específicos}

\begin{itemize}
	\item Armazenar dados para levantamentos estatísticos;
	\item Implementar o sistema de identificação e controle de usuários;
	\item Implementar o sistema de interface;
	\item Implementar o sistema de atuação;
	\item Implementar o sistema de emulação de dados;
	\item Integrar todos os sistemas a fim de garantir as funcionalidades do projeto;
	\item Monitorar em ambiente online dos parâmetros do sistema;
\end{itemize}

\section{Justificativas e Relev{\^a}ncia}
%
Segundo \citeauthor{AlvesDaSilva} (\citeyear{AlvesDaSilva}), o crescente consumo de água
tem feito do uso consciente uma necessidade primordial. Essa prática deve ser considerada parte de uma atividade mais abrangente que é o uso racional da água, incluindo também, o controle de perdas e a redução do consumo de água.

Ao passar dos anos, os desperdícios da água utilizada atingem níveis nunca imaginados \cite{rebouccas2003agua}. Ao combinar as técnicas de eletrônica e automação, será possível otimizar o monitoramento do gasto de água em residências e prédios, visando coletar dados para diminuir este tipo de desperdício. Este trabalho se justifica pela urgente necessidade de controle do uso da água em todas as esferas da sociedade. 
%
\section{Organização do trabalho}

O presente trabalho está organizado em 5 Capítulos. No Capitulo 1, encontra-se a apresentação do problema e suas possíveis soluções, além de apresentar os objetivos propostos.

O Capítulo 2 consiste na revisão sobre os hardwares e softwares utilizados no projeto, encontram-se as definições e explicações dos mesmos.

No Capítulo 3 é apresentado a estrutura geral do sistema, a metodologia em que foi construído. Explica-se também as etapas do desenvolvimento e códigos implementados.

O Capítulo 4 trata dos experimentos realizados no sistema e seus resultados.

No Capítulo 5 são abordadas as considerações finais do trabalho e os possíveis trabalhos futuros.